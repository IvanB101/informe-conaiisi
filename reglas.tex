\section{Afiliaciones y cálculo de cuotas}

Según lo establecido en \cite{CSOrd53}, además del personal docente y no docente de la \acrshort{unsl}, otras personas que cumplan con las condiciones dictadas también pueden afiliarse a \acrshort{dospu}.

A raíz de esto, los afiliados se encuentran dividos en las categorías titular (\cref{sec:titular}), familiar (\cref{sec:familiar}) y voluntario adherente (\cref{sec:adherente}). Asimismo cada categoría está dividida en subcategorías, cada una utilizando distintas fórmulas o coeficientes en las mismas para el cálculo del aporte del afiliado. Las definciones de cada categoría y subcategoría se pueden encontrar en \cite{CSOrd53} art. 24.

\subsection{Titular} \label{sec:titular}
Dentro de la categoría titular se distingue entre obligatorio activo y voluntario jubilado.

\subsubsection{Obligatorio activo} 
Actualmente, este descuento no es cálculado por el \acrshort{si}, con lo cual está fuera del alcance de este trabajo.

\subsubsection{Voluntario jubilado}
Monto de la cuota (\cite{dospuRes21} art. 2 y Anexo I): 
\begin{displaymath}
0.02 * j_m + 0.05 * j_h
\end{displaymath}

, donde:
\begin{itemize}
    \item $j_m = \text{jubilación mínima}$
    \item $j_h = \text{haber jubilatorio}$
\end{itemize}

En caso de que dos voluntarios jubilados se tengan vínculo de cónyuge o conviviente, se aplica un descuento del 30\% a la cuota del afiliado (cálculada con la fórmula expuesta para esta subcategoría) con menor haber percibido, quedando como 70\% del monto calculado.

\subsection{Familiar} \label{sec:familiar}
Los familiares de obligatorios activos son no aportantes.

Por otra parte, para cónyuges y convivientes de voluntarios jubilados, su cuota es un 70 \% de la del afiliado titular (\cite{dospuRes21} art. 2 y Anexo I).

\subsection{Voluntario adherente} \label{sec:adherente}
Dentro de esta categoría, se distingue entre las subcategorías: pensionado, becarios y personal ad honorem de la \acrshort{unsl}, agente \acrshort{unsl} con licencia, ascendientes en primer grado (de un afiliado titular), hijos (que hayan dejado de reunir las condiciones de no aportantes), familiares adherentes, universitarios adherentes, ex-afiliados a \acrshort{dospu}, agentes vinculados a \acrshort{dospu} y adherente de edad avanzada.

Según \cite{dospuRes21} art. 3 para los afiliados de esta categoría, a excepción de las subcategorías tratadas en los \crefrange{sssec:pensionado}{sssec:edad_avanzada}, el monto de la cuota se calcula como un porcentaje sobre el valor de referencia \acrshort{cmmu}. Los porcentajes se encuentran en el Anexo II de la referencia mencionada.

La \acrfull{cmmu} se define en el 6 \% del sueldo total bruto de un Profesor Universitario Titular Exclusivo con Máxima Antigüedad (\cite{dospuRes21} art. 3).

\subsubsection{Pensionado}\label{sssec:pensionado}
Monto de la cuota (\cite{dospuRes21} art. 2 y Anexo I):
\begin{displaymath}
0.02 * j_m + 0.05 * p
\end{displaymath}

, donde:
\begin{itemize}
    \item $j_m = \text{jubilación mínima}$
    \item $p = \text{pensión}$
\end{itemize}

\subsubsection{Ascendiente en primer grado}
Para un afiliado de esta categoría con más de diez (10) años de antigüedad en DOSPU, se utiliza como valor de referencia \acrshort{cmmu}20, equivalente a 6\% de un Profesor Universitario Titular Exclusivo con una Antigüedad correspondiente veinte años, en lugar de la \acrshort{cmmu}\cite{dospuRes60}.

Asimismo, para afiliados mayores a 66 años de edad, se toma el 150\% en lugar de 200\% del valor de referencia, siendo este último el utilizado para los ascendientes en primer grado sin la antigüedad requerida.

En caso de no contar con dichos aportes, el valor de la cuota se calcula como un porcentaje sobre el valor de referencia \acrshort{cmmu} (\cite{dospuRes21} art. 3).

\subsubsection{Adherentes de edad avanzada}\label{sssec:edad_avanzada}
El monto a abonar por los afiliados pertenecientes a esta subcategoría depende de si el afiliado tiene 25 o más años de aporte de \acrfull{decom} (\cite{dospuRes7} art. 1.d.):
\begin{itemize}
    \item En caso de tener dicho aporte se toma un 150\% de la \acrshort{cmmu}
    \item En caso contrario se toma un 200\% de la \acrshort{cmmu}.
\end{itemize}

\subsection{Aportes y seguros adicionales}
Adicionalmente, dependiendo de la categoría y subcategoría del afiliado, el monto de la cuota pued incluir aportes a \acrfull{fesac} y \acrfull{sumas}, así como un seguro en caso de fallecimiento \cite{dospuRes31, dospuRes43, dospuRes71}.

\subsection{Modificadores de afiliación}
Para afiliados voluntarios adherente que ingresen con enfermedades preexistentes o que excedan la edad de 65 años se aplica un modificador al monto de la cuota cálculada, el cual depende del carácter de la enferdad del afiliado (ver \cref{tbl:modificadores}).

\begin{table}
\begin{tabular}{|c|c|}
    \hline
    Carácter de la enfermedad & Modificador \\ \hline
    Temporario & 2 \\ \hline
    Crónico & 3 \\ \hline
    De mayor complejidad & 2 \\ \hline
\end{tabular}
\caption{Modificadores de afiliación}
\label{tbl:modificadores}
\end{table}

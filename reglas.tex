\section{Afiliaciones y cálculo de cuotas}\label{sec:afiliaciones}

Según lo establecido en \cite{CSOrd53}, además del personal docente y no docente de la \acrshort{unsl}, otras personas también pueden afiliarse a \acrshort{dospu}.
%
A raíz de esto, los afiliados se encuentran dividos en las categorías titular, familiar y voluntario adherente. Asimismo cada categoría está dividida en subcategorías, cada una utilizando distintas fórmulas o coeficientes en las mismas para el cálculo del aporte del afiliado  \cite{dospuRes21}. 
% Las definciones de cada categoría y subcategoría se pueden encontrar en \cite{CSOrd53} art. 24.

\subsection{Titular} 
\label{sec:titular}

La categoría titular tiene dos subcategorías.
La primera es \textbf{Obligatorio Activo}, cuyo aporte es calculado fuera del \acrshort{si}, por lo que es excluido del alcance de este trabajo.

% \subsubsection{Voluntario jubilado} \label{sssec:jubilado}

La segunda es \textbf{Voluntario Jubilado}, siendo el monto de su cuota: 
\begin{displaymath}
0.02 * j_m + 0.05 * j_h
\end{displaymath}
donde $j_m$ es la \text{jubilación mínima} determinada por ley y $j_h$ es el \text{haber jubilatorio} percibido.
%
En caso de que dos voluntarios jubilados se tengan vínculo de cónyuge o conviviente, se aplica un descuento del 30\% a la cuota del afiliado con menor haber percibido.
%
Adicionalmente, el valor de esta cuota no puede ser inferior a un valor de referencia (\scode{cuotaMinimaJubilado}).
%
Por último, en caso de no tener un haber percibido cargado en el sistema, se utiliza otro valor de referencia (\scode{cuotaMaximaJubilado}) como el monto de la misma.

\subsection{Familiar} 
\label{sec:familiar}
Los familiares de obligatorios activos son no aportantes.
%
Por otra parte, para cónyuges y convivientes de voluntarios jubilados, su cuota es un 70 \% de la del afiliado titular.

\subsection{Voluntario adherente} \label{sec:adherente}

Esta categoría distingue las subcategorías:
\begin{enumerate*}[label=(\alph*)]
    % \item 
    % \label{item:pensionado}
    % pensionado, 
    \item becarios y personal ad honorem de la \acrshort{unsl}, 
    \item agente \acrshort{unsl} con licencia, 
    \item 
    \label{item:ascendientes}
    ascendientes en primer grado (de un afiliado titular), 
    \item hijos (que hayan dejado de reunir las condiciones de no aportantes), 
    \item familiares adherentes, 
    \item universitarios adherentes, 
    \item ex-afiliados a \acrshort{dospu}, 
    \item agentes vinculados a \acrshort{dospu} y 
    \item 
    \label{item:avanzada}
    adherente de edad avanzada.
\end{enumerate*} 

El monto de la cuota para los afiliados de esta categoría se calcula como un porcentaje sobre el valor de referencia \acrfull{cmmu}, la cual se define como el 6 \% del sueldo total bruto de un Profesor Universitario Titular Exclusivo con Máxima Antigüedad. %(\cite{dospuRes21} art. 3).
% La excepción a esta regla son las subcategorías 
% \ref{item:pensionado}, 
% \ref{item:ascendientes} y 
% \ref{item:avanzada}.
La excepción a esta regla es la subcategoría \ref{item:ascendientes}.

% \subsubsection{Pensionado}
% \label{sssec:pensionado}

% Para el pensionado \ref{item:pensionado}, el monto de la cuota se calcula como:
% \begin{displaymath}
% 0.02 * j_m + 0.05 * p
% \end{displaymath}
% donde $p$ es el monto percibido como pensión.

% \subsubsection{Ascendiente en primer grado}
En el caso ascendientes en primer grado \ref{item:ascendientes} se distinguen dos casos.
Si el afiliado cuenta con más de diez (10) años de antigüedad en DOSPU, se utiliza como valor de referencia \acrshort{cmmu}20, equivalente a 6\% de un Profesor Universitario Titular Exclusivo con una Antigüedad correspondiente veinte años, en lugar de la \acrshort{cmmu}\cite{dospuRes60}.

% Asimismo, para afiliados mayores a 66 años de edad, se toma el 150\% en lugar de 200\% del valor de referencia, siendo este último el utilizado para los ascendientes en primer grado sin la antigüedad requerida.

% En caso de no contar con dichos aportes, el valor de la cuota se calcula como un porcentaje sobre el valor de referencia \acrshort{cmmu} (\cite{dospuRes21} art. 3).

% \subsubsection{Adherentes de edad avanzada}
% \label{sssec:edad_avanzada}

% El monto a abonar por los adherentes de edad avanzada \ref{item:avanzada} depende de si el afiliado tiene 25 o más años de aporte de \acrfull{decom} \cite{dospuRes7}.
% En caso de tener dicho aporte se toma un 150\% de la \acrshort{cmmu}. 
% En caso contrario se toma un 200\% de la \acrshort{cmmu}.


% \subsection{Aportes y seguros adicionales}

% Adicionalmente, dependiendo de la categoría y subcategoría del afiliado, el monto de la cuota pued incluir aportes a \acrfull{fesac} y \acrfull{sumas}, así como un seguro en caso de fallecimiento \cite{dospuRes31, dospuRes43, dospuRes71}.

% \subsection{Modificadores de afiliación}
% Para afiliados voluntarios adherente que ingresen con enfermedades preexistentes o que excedan la edad de 65 años se aplica un modificador al monto de la cuota cálculada, el cual depende del carácter de la enferdad del afiliado (ver \cref{tbl:modificadores}).

% \begin{table}
% \begin{tabular}{|c|c|}
%     \hline
%     Carácter de la enfermedad & Modificador \\ \hline
%     Temporario & 2 \\ \hline
%     Crónico & 3 \\ \hline
%     De mayor complejidad & 2 \\ \hline
% \end{tabular}
% \caption{Modificadores de afiliación}
% \label{tbl:modificadores}
% \end{table}

\section{Afiliaciones y cálculo de cuotas}\label{sec:afiliaciones}

Según lo establecido en \cite{CSOrd53}, además del personal docente y no docente de la \acrshort{unsl}, otras personas también pueden afiliarse a \acrshort{dospu}.
%
A raíz de esto, los afiliados se encuentran divididos en las categorías titular, familiar y voluntario adherente. Asimismo, cada categoría está dividida en subcategorías, cada una utilizando distintas fórmulas o coeficientes en las mismas para el cálculo del aporte del afiliado  \cite{dospuRes21}. 

\subsection{Titular} 
\label{sec:titular}

La categoría titular tiene dos subcategorías.
La primera es \textbf{Obligatorio Activo}, cuyo aporte es calculado fuera del \acrshort{si}, por lo que es excluido del alcance de este trabajo.

La segunda es \textbf{Voluntario Jubilado}, siendo el monto de su cuota: 
\begin{displaymath}
0.02 * j_m + 0.05 * j_h
\end{displaymath}
donde $j_m$ es la \text{jubilación mínima} determinada por ley y $j_h$ es el \text{haber jubilatorio} percibido.
%
En caso de que dos voluntarios jubilados sean cónyuges o convivientes, se aplica un descuento del 30\% a la cuota del afiliado con menor haber percibido.
%
Adicionalmente, el valor de esta cuota no puede ser inferior a la \textit{cuota mínima de jubilado}, un valor tomado como referencia.
%
Por último, en caso de no tener un haber percibido actualizado, se utiliza otro valor de referencia, llamado \textit{cuota máxima de jubilado}, como el monto de la misma.

\subsection{Familiar} 
\label{sec:familiar}
Los familiares de obligatorios activos son no aportantes.
%
Por otra parte, para cónyuges y convivientes de voluntarios jubilados, su cuota es un 70 \% de la del afiliado titular.

\subsection{Voluntario adherente} \label{sec:adherente}

Esta categoría distingue las subcategorías:
\begin{enumerate*}[label=(\alph*)]
    % \item 
    % \label{item:pensionado}
    % pensionado, 
    \item becarios y personal ad honorem de la \acrshort{unsl}, 
    \item agente \acrshort{unsl} con licencia, 
    \item 
    \label{item:ascendientes}
    ascendientes en primer grado (de un afiliado titular), 
    \item hijos (que hayan dejado de reunir las condiciones de no aportantes), 
    \item familiares adherentes, 
    \item universitarios adherentes, 
    \item ex-afiliados a \acrshort{dospu}, 
    \item agentes vinculados a \acrshort{dospu} y 
    \item 
    \label{item:avanzada}
    adherente de edad avanzada.
\end{enumerate*} 

El monto de la cuota para los afiliados de esta categoría se calcula como un porcentaje sobre el valor de referencia \acrfull{cmmu}, la cual se define como el 6 \% del sueldo total bruto de un Profesor Universitario Titular Exclusivo con Máxima Antigüedad. 
La excepción a esta regla es la subcategoría \ref{item:ascendientes}.

En el caso ascendientes en primer grado \ref{item:ascendientes} se distinguen dos casos.
Si el afiliado cuenta con más de diez (10) años de antigüedad en DOSPU, se utiliza como valor de referencia \acrshort{cmmu}20, equivalente a 6\% de un Profesor Universitario Titular Exclusivo con una Antigüedad correspondiente a veinte años, en lugar de la \acrshort{cmmu}\cite{dospuRes60}.

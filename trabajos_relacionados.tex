\section{Trabajos Relacionados}\label{sec:trabajos_relacionados}

Si bien no se encontró un caso con características tan similares, si se encontraron casos de sistemas informáticos adoptando motores de reglas para facilitar su adaptación a cambios en las reglamentaciones. 
A continuación describimos brevemente algunos de ellos.

En \cite{medic2019calculation} se reporta un uso similar de un \acrshort{brms} \cite{proctor2012drools} para el cálculo de los precios de las distintas pólizas de una aseguradora, siendo el principal objetivo buscado la mantenibilidad y la adaptabilidad a cambios en las regulaciones.

En \cite{sampol2019sistema}, en el contexto nacional, se describe el uso de un \acrshort{brms} en aplicaciones desarrolladas para SENASA con el objetivo de lograr que tengan mayor flexibilidad frente a cambios en las reglas de negocio que rigen los proceso que soportan.

También se detectó que en sistemas informáticos para organismos de salud privados, se argumenta flexibilidad en la especificación de reglas de negocio, pero sin mayor detalle que el comercial.

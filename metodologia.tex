\section{Metodología} \label{sec:metodologia}

El enfoque adoptado combina pasos de metodologías de desarrollo de software y de metodologías de investigación científica, poniendo en juego conocimiento y habilidades adquiridos en el cursado de la Ingeniería en Informática de la \acrshort{unsl} en cada tarea ejecutada.

\revisar{No me queda claro que sería lo de establecer oportunidades (de mejora, de clarificación?)}
Se partió de la lectura y estudio de las ordenanzas y documentos detallando aspectos relevantes de \acrshort{dospu} y las fórmulas utilizadas para el cálculo de las cuotas de sus afiliados, que fue complementada con reuniones con miembros del equipo de desarrollo {\SIDOSPU} para despejar dudas y/o establecer oportunidades. 
De igual forma, se indagó sobre las tecnologías utilizadas en el sistema y la materialización de las reglas de negocio implementadas en el mismo.

Se extendieron los casos de prueba ya utilizados en el ciclo de integración continua de {\SIDOSPU}.
Estos nuevos casos se definieron en base a la información obtenida, utilizando una combinación de \acrfull{avl} y casos Ad Hoc para capturar el comportamiento esperado en un nivel de granularidad más fina y poder comprobarlo en cada paso ejecutado.

Adicionalmente, durante el estudio del código fuente encargado del cálculo de las cuotas, se detectaron oportunidades de mejora. 
En consecuencia, como paso intermedio, para mejorar la calidad del código y tener un mejor punto de partida para la posterior comparación con el motor de reglas, se realizó un refactorizado de la parte del sistema correspondiente. 
% Esta refactorización consistió principalmente en el uso de patrones de programación funcional para el manejo de errores monádico.
Esta refactorización consistió principalmente en el uso de patrones de programación funcional para el manejo de errores, haciendo uso de mónadas.

Por otro lado, se relevaron motores de reglas que se ajusten a las necesidades de expresividad requeridas para la materialización de las reglas de negocio de una \acrshort{osu}, y a las tecnologías de implementación del {\SIDOSPU}. 
Partiendo de este relevamiento, se realizó la selección del motor que se consideró más apropiado.

Luego se incorporó el motor de reglas en el {\SIDOSPU}.
% Seguidamente, se reimplementó parte del sistema recurriendo al motor de reglas para separar la especificación de las reglas de negocio del código fuente.
Seguidamente, se realizó la especificación de las reglas de negocio utilizando el formato requerido por el motor de reglas.

Esta nueva versión del cálculo de las cuotas convive con la mejorada y con la original. 
Esto se consiguió definiéndolas como implementaciones de una única interfaz Java y múltiples perfiles de Spring. 
Esto facilitó el análisis de resultados obtenidos y verificaciones a partir de casos de prueba. 

% Para la comparación de entre las implementaciones se utilizó como métrica la cantidad de líneas de código.  
Para la comparación entre las implementaciones se utilizó como métrica la cantidad de líneas de código.  
En el caso de la implementación con el motor de reglas se utilizan en su lugar las filas de las tablas en las que se especifican las reglas.

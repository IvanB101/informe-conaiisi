\documentclass[12pt,a4paper,spanish]{article} 
\usepackage[spanish]{babel}

\setlength{\parindent}{0pt}
\usepackage{sectsty}
\sectionfont{\rmfamily\fontsize{12pt}{14.4pt}\selectfont}
\usepackage{titlesec}
\titlespacing*{\section} {0pt}{0ex}{0ex}
\pagestyle{empty}

\usepackage[bookmarks=true]{hyperref}

\usepackage{ragged2e}
\usepackage[none]{hyphenat}

% Agrega toc al toc
\usepackage{tocbibind}

\usepackage[acronym]{glossaries}
\makeglossaries

\newacronym{si}{SI}{Sistema Informático}
\newacronym[
longplural=Obras Sociales Universitarias
]{osu}{OSU}{Obra Social Universitaria}
\newacronym{dospu}{DOSPU}{Dirección de Obra Social para el Personal Universitario}
\newacronym{decom}{DECOM}{Departamento de Complementación}
\newacronym{fesac}{FESAC}{Fondo Especial Solidario para Alta Complejidad}
\newacronym{sumas}{SUMAS}{Sistema Universitario Médico Asistencial Solidario}
\newacronym{unsl}{UNSL}{Universidad Nacional de San Luis}
\newacronym{cmmu}{CMMU}{Cuota Mensual Máxima Única}
\newacronym{cmmu20}{CMMU20}{Cuota Mensual Máxima Única 20} % TODO: agregar descripcion
\newacronym{js}{JS}{Javascript}
\newacronym{ts}{TS}{Typescript}
\newacronym{so}{SO}{Sistema Operativo}
\newacronym{avl}{AVL}{Análisis del Valor Límite}
\newacronym{onu}{ONU}{Organización de las Naciones Unidas}
\newacronym[
	user1={Bussiness Rule Management Service}
]{brms}{BRMS}{Sistema de Gestión de Reglas de Negocio}
\newacronym[
	user1={Decision Requirements Diagram}
]{drd}{DRD}{Diagrama de Requemientos de Decisión}
\newacronym[
	user1={Drools Rule Language}
]{drl}{DRL}{Lenguaje de Reglas de Drools}
\newacronym[
	user1={Friendly Enough Expression Language}
]{feel}{FEEL}{Lenguaje de Expresión Suficientemente Amigable}
\newacronym[
	user1={Decision Model Notation}
]{dmn}{DMN}{Notación de Modelo de Decisión}
\newacronym[
	user1={Bussiness Expression}
]{bex}{BEX}{Expresión de Negocios}
\newacronym[ user1={Plain Old Java Object} ]{pojo}{POJO}{Objeto Java Simple}
\newacronym[
	user1={Database Management System}
]{dbms}{DBMS}{Sistema de Gestión de Base de Datos}
\newacronym[
	user1={Atomicity, Consistency, Isolation and Durability}
]{acid}{ACID}{Atomicidad, Consistencia (o Integridad), Aislamiento y Duranbilidad (o Persistencia)}
\newacronym[
	user1={Aspect Oriented Programming}
]{aop}{AOP}{Programación Orientada a Aspectos}
\newacronym[ user1={Model View Controller} ]{mvc}{MVC}{Modelo Vista Controlador}
\newacronym[ user1={Model View Viewmodel} ]{mvvm}{MVVM}{Modelo Vista Modelo de Vista}
\newacronym[
	user1={Representational Data Transfer}
]{rest}{REST}{Transferencia de Estado Representacional}
\newacronym[
	user1={Java Database Connectivity}
]{jdbc}{JDBC}{Conectividad a Base de Datos de Java}
\newacronym[ user1={Object Relational Mapping} ]{orm}{ORM}{Mapeo Relacional de Objectos}
\newacronym[ user1={Object XML Mapping} ]{oxl}{OXL}{Mapeo de Objectos a XML}
\newacronym[ user1={Java Message Service} ]{jms}{JMS}{Servicio de Mensajes de Java}
\newacronym[ user1={Project Object Model} ]{pom}{POM}{Modelo de Objetos de Proyecto}
\newacronym[ user1={Java Archive} ]{jar}{JAR}{Archivo Java}
\newacronym[ user1={Web Application Resource} ]{war}{WAR}{Recurso de Aplicación Web}
\newacronym[ user1={Java Virtual Machine} ]{jvm}{JVM}{Máquina Virtual de Java}
\newacronym[
	user1={HyperText Markup Language}
]{html}{HTML}{Lenguaje de Marcardo de Hipertexto}
\newacronym[
	user1={Estensible HyperText Markup Language}
]{xhtml}{XHTML}{Lenguaje de Marcardo de Hipertexto Extensible}
\newacronym[ user1={Extensible Markup Language} ]{xml}{XML}{Lenguage de Marcado Extensible}
\newacronym[ user1={Scalable Vector Graphics} ]{svg}{SVG}{Gráficos Vectoriales Escalables}
\newacronym[
    user1={Mathematical Markup Language}
]{mathml}{MathML}{Lenguage de Marcado Matemático}
\newacronym[ user1={Cascading Style Sheet} ]{css}{CSS}{Hojas de Estilo en Cascada}
\newacronym[ user1={Single Page Application} ]{spa}{SPA}{Aplicación de página única}
\newacronym[ user1={Document Object Model} ]{dom}{DOM}{Modelo de Objetos del Documento}
\newacronym[ user1={Static Site Generation} ]{ssg}{SSG}{Generación Estática de Sitios}
\newacronym[ user1={Server Side Rendering} ]{ssr}{SSR}{Renderizado del Lado del Servidor}
\newacronym[ user1={World Wide Web Consortium} ]{w3c}{W3C}{Consorcio WWW}
\newacronym[ user1={Just In Time} ]{jit}{JIT}{Justo a tiempo}
\newacronym[ user1={Language Server Protocol} ]{lsp}{LSP}{Protocolo de Servidor de Lenguage}
\newacronym[ user1={Test Driven Development} ]{tdd}{TDD}{Desarrollo Guiado por Pruebas}
\newacronym[
    user1={Behavior Driven Development}
]{bdd}{BDD}{Desarrollo Guiado por Comportamiento}
\newacronym[
    user1={Application Programming Interface}
]{api}{API}{Interfaz de Programación de Aplicaciones}
\newacronym[ user1={Continous Delivery} ]{cd}{CD}{Entrega Continua}
\newacronym[ user1={Continous Integration} ]{ci}{CI}{Integración Continua}


\usepackage[inline]{enumitem}

\usepackage{cleveref}
\crefname{listing}{listado}{listados}
\Crefname{listing}{Listado}{Listados}

\usepackage{multicol}

\usepackage{fontspec}
\setmonofont{MesloLGS NF}
\setmainfont{Times New Roman}

% \usepackage{tikz}
% \usetikzlibrary{arrows.meta, positioning}

\usepackage[margin=2.5cm]{geometry}

% \usepackage{minted}
% \setminted[]{
%     breaklines,
%     breakafter=(.,
%     fontsize=\scriptsize,
%     escapeinside=||,
%     frame=single,
%     rulecolor=\color{black},
%     tabsize=2,
% }
% \setmintedinline[]{
%     fontsize=\scriptsize,
% }

% \usepackage{graphicx}
% \graphicspath{ {./images/} }

% \usepackage{csquotes}
\usepackage[backend=biber, style=alphabetic, sorting=none, bibstyle=alphabetic, maxnames=10, minnames=6,backref=true, autocite=inline, labelalpha=true, doi=false, url=false]{biblatex}
\addbibresource{the.bib}

% Use more than one optional parameter in a new commands
\usepackage{xargs}                      

% \usepackage[spanish,colorinlistoftodos,prependcaption,textsize=tiny]{todonotes}
% \newcommandx{\desarrollar}[2][1=]{\todo[inline,linecolor=blue,backgroundcolor=blue!25,bordercolor=blue,#1]{#2}}
% \newcommandx{\modificar}[2][1=]{\todo[linecolor=Plum,backgroundcolor=Plum!25,bordercolor=Plum,#1]{#2}}
% \newcommandx{\cambiarpor}[2][1=]{\todo[inline,linecolor=Plum,backgroundcolor=Plum!25,bordercolor=Plum,#1]{Cambiar $\uparrow$ por:\\#2}}
% \newcommandx{\revisar}[2][1=]{\todo[linecolor=red,backgroundcolor=red!25,bordercolor=red, inline,#1]{#2}}
% \newcommandx{\agregaren}[2][1=]{\todo[inline,linecolor=Plum,backgroundcolor=Plum!25,bordercolor=Plum,#1]{Agregar punto seguido $\uparrow$:\\#2}}
% \newcommandx{\agregar}[2][1=]{\todo[inline,linecolor=Plum,backgroundcolor=Plum!25,bordercolor=Plum,#1]{Agregar:\\#2}}
% \newcommandx{\info}[2][1=]{\todo[inline,linecolor=OliveGreen,backgroundcolor=OliveGreen!25,bordercolor=OliveGreen,#1]{#2}}
% \newcommandx{\eliminar}[2][1=]{\todo[inline,linecolor=red,backgroundcolor=red!25,bordercolor=red,#1]{\textbf{Eliminar $\uparrow$:}\\#2}}

% \newcommandx{\improvement}[2][1=]{\todo[linecolor=Plum,backgroundcolor=Plum!25,bordercolor=Plum,#1]{#2}}
% \newcommandx{\thiswillnotshow}[2][1=]{\todo[disable,#1]{#2}}

\newcommand{\SIOSU}{\acrshort{si}--\acrshort{osu}}
\newcommand{\SIDOSPU}{\acrshort{si}--\acrshort{dospu}}

\newcommand{\CARRERA}{Ingeniería en Informática de la \acrfull{unsl}}

\newcommand*{\thistitle}{
\twocolumn[
\bgroup
	\centering
    {\fontsize{16pt}{17.2pt}\selectfont\bfseries Parametrizando un Sistema Informático para Obra Social Universitaria con un Motor de Reglas \par}
	\vspace{0.5cm}
  %   {\fontsize{14pt}{16.8pt}\selectfont\bfseries Iván Brocas (autor), Alejandro Sánchez (tutor), Carlos H. Salgado (tutor)  \par}
	% {\fontsize{12pt}{14.4pt}\selectfont\bfseries\itshape Universidad Nacional de San Luis, Facultad de Ciencias Físico Matemática y Naturales \par}
	\vspace{0.5cm}
\egroup
]
}

\newcommand{\code}[1]{{\texttt{#1}}}
\newcommand{\scode}[1]{{\footnotesize\code{#1}}}
\newcommand{\tblcode}[1]{{\sffamily\small{#1}}}


\let\originalacrfull\acrfull
\RenewDocumentCommand{\acrfull}{m}{%
  \ifglshasfield{user1}{#1}{%
    \glsentrylong{#1} (\glsentryuseri{#1}, \glsentryshort{#1})%
  }{%
    \originalacrfull{#1}%
  }%
}

\newcommand{\CenteredGraphic}[2]{%
	{\centering \resizebox{#2\textwidth}{!}{\includegraphics{#1}} \\}
}

\defbibheading{bibliography}[\refname]{%
  \section*{\fontsize{10pt}{12pt}\selectfont #1}%
}
\renewcommand*{\bibfont}{\fontsize{10pt}{12pt}\selectfont}
\urlstyle{rm} 
\DeclareFieldFormat{title}{\textnormal{#1}}
\DeclareFieldFormat{journaltitle}{\textnormal{#1}}
\DeclareFieldFormat{booktitle}{\textnormal{#1}}



\renewenvironment{abstract}{
  \fontsize{10pt}{12pt}\selectfont
  {\bfseries Abstract}

  \itshape
}{
 \par\vspace{1em}
}

\newenvironment{keywords}{
  \fontsize{10pt}{12pt}\selectfont
  {\bfseries Palabras Clave}

}{
 \par\vspace{1em}
}


\title{Parametrizando un Sistema Informático para Obra Social Universitaria con un Motor de Reglas}
\author{Iván Brocas}

\begin{document}

\thistitle

\setlength\columnsep{1cm}
\begin{multicols}{2}
    \justifying
    \begin{abstract}
Este trabajo describe la parametrización de un sistema informático de una obra social universitaria. 
Esto consiste en la extracción de las reglas de negocio del código fuente del sistema, con el fin de separar la gestión de las mismas. 
Con esto se busca lograr un sistema más mantenible que pueda responder rápidamente a modificaciones en las reglamentaciones que rigen su funcionamiento, pudiendo aplicar cambios a las reglas sin necesidad de cambiar el código fuente y/o volver a desplegar el sistema. 
En un contexto socioeconómico donde la única constante es el cambio, esto facilitaría que el sistema se mantenga actualizado, evitando la obsolescencia y conservando el valor que brinda a la organización. 
Se hace foco en las reglas que rigen el cálculo de las cuotas de los afiliados. 
Esto se debe a que la obra social estudiada busca equidad en la distribución de la carga de los ingresos económicos, lo que lleva a plantear diversos tipos de afiliación y fórmulas de cálculo de cuota asociadas. 
Para lograr la separación entre las reglas de negocio y el sistema informático, se hace uso de un motor de reglas, el cual es integrado con el sistema existente.
\end{abstract}


    \begin{keywords}
    Obra social universitaria, motor de reglas, sistema informático, OpenL Tablets, cálculo de cuotas
    \end{keywords}

    \section{Introducción}\label{sec:intro}

Este proyecto integrador final se encuentra enmarcado en el contexto del desarrollo de un \acrfull{si} para \acrfullpl{osu} caracterizado por la centricidad en el afiliado y la agilidad. 
Este trabajo se centra en la segunda, procurando reducir costos y tiempos de adaptación del sistema a cambios a partir de la introducción de un motor de reglas que permita separar del código fuente las reglas del negocio tendientes a cambiar.

El \acrshort{si} de una \acrshort{osu} debe mantenerse ``vivo''. 
El valor que provee su funcionamiento disminuye conforme pierde sintonía con los cambios que tanto la organización 
como su contexto sufren. 
Dado que el cambio es la norma, no la excepción, la falta sistemática de evolución de un {\SIOSU} representa su agonía y eventual muerte, la cual es generalmente acompañada de numerosos perjuicios para la \acrshort{osu}.

Los funcionarios, al encontrarse con un \acrshort{si} desactualizado, suelen responder por medio de trabajo manual. Conforme la brecha entre las reglas de negocio, dictadas por las reglamentaciones, y la implementación de las mismas en el \acrshort{si} crece, también lo hace la porción manual del trabajo. Esto resulta en más trabajo para el personal y trámites más lentos, lo que causa mayor descontento de los afiliados. De la misma forma, el decremento en la porción del trabajo realizada por el sistema se traduce en menos información para el respaldo de decisiones.

La \acrfull{dospu} de la \acrfull{unsl} no escapa a este problema. 
Su presidencia, el Rectorado, la Facultad de Ciencias Físico-Matemáticas y Naturales, y su Departamento de Informática se encuentran colaborando para reemplazar su antiguo \acrshort{si}, que ha fallado en evolucionar. 
%
El nuevo sistema, \acrshort{si}-\acrshort{dospu}, está siendo desarrollado usando prácticas de las metodologías ágiles: scrum, product discovery, behaviour driven development, test driven development, e integración continua.
%
Las tecnologías de implementación incluyen el lenguaje de programación Java, el framework Spring \cite{springframework} y la base de datos PostgreSQL \cite{postgresql} para el backend, y Angular \cite{angular} para el frontend. 

En un trabajo previo, documentado en \cite{Vela2024}, el foco estuvo sobre implementar el expendio de órdenes procurando la centricidad en el paciente.
%
Este proyecto integrador aborda la agilidad requerida por el {\SIOSU}, entendida como la facilidad de adaptación a cambios en el contexto socio-económico y en las reglamentaciones que regulan su funcionamiento.

El objetivo general es parametrizar {\SIDOSPU}, introduciendo un mecanismo que permita separar las reglas de negocio de la organización, de su \acrshort{si}. 
Se detectó que la reglamentación para el cálculo de la cuota de afiliados es compleja y muy susceptible a cambios en la realidad socio-económica del país. 
Esto se debe a que la obra social busca equidad en la distribución de la carga de los aportes, lo que lleva a plantear diversos tipos de afiliación y fórmulas de cálculo de cuota. 
Estas características convierten a dicha funcionalidad en candidata ideal para enfocar el esfuerzo.

El mecanismo típico para lograr esta separación es un motor de reglas. 
Este permite la ejecución de reglas de negocio, especificadas fuera del código del sistema, utilizando un lenguaje o formato específico.

Del objetivo general se desprenden cuatro objetivos específicos:
\begin{enumerate}
    \item \label{obj:esp:extraer}
    Conocer las reglas del negocio correspondientes al cálculo de la cuota de afiliación a partir de las reglamentaciones pertinentes y su materialización en el código fuente del {\SIDOSPU}.
    \item \label{obj:esp:intelegible}
    Expresar dichas reglas en un lenguaje que resulte entendible para el personal de la \acrlong{osu}.
    \item \label{obj:esp:independiente}
    Permitir la gestión de las especificaciones obtenidas de forma independiente al \acrshort{si}.
    \item \label{obj:esp:esfuerzo}
    Reducir el esfuerzo requerido para materializar cambios en las reglas del negocio, sin requerir modificar el código fuente del sistema, ni realizar el despliegue de una nueva versión.
\end{enumerate}

El logro de estos objetivos permitirá que cambios en las reglas de negocio puedan materializarse a través de la edición de especificaciones externas,  permitiendo la reducción en la incidencia a problemas mencionados anteriormente.

\section{Trabajos Relacionados}\label{sec:trabajos_relacionados}

Si bien no se encontró un caso con características tan similares, si se encontraron casos de sistemas informáticos adoptando motores de reglas para facilitar su adaptación a cambios en las reglamentaciones. 
A continuación describimos brevemente algunos de ellos.

En \cite{medic2019calculation} se reporta un uso similar de un \acrshort{brms} \cite{proctor2012drools} para el cálculo de los precios de las distintas pólizas de una aseguradora, siendo el principal objetivo buscado la mantenibilidad y la adaptabilidad a cambios en las regulaciones.

En \cite{sampol2019sistema}, en el contexto nacional, se describe el uso de un \acrshort{brms} en aplicaciones desarrolladas para SENASA con el objetivo de lograr que tengan mayor flexibilidad frente a cambios en las reglas de negocio que rigen los proceso que soportan.

También se detectó que en sistemas informáticos para organismos de salud privados, se argumenta flexibilidad en la especificación de reglas de negocio, pero sin mayor detalle que el comercial.


El resto de este trabajo se encuentra dividido en:
el \cref{sec:metodologia} explica la metodología utilizada; 
el \cref{sec:motores} enumera los motores de reglas considerados y el razonamiento detrás de la selección realizada dentro de las opciones;
el \cref{sec:afiliaciones} trata las distintas categorías de afiliaciones de la \acrshort{osu} y el cálculo de sus cuotas;
el \cref{sec:integracion} explica cómo se realizó la integración del motor de reglas seleccionado con el {\SIOSU};
el \cref{sec:resultados} expone los resultados obtenidos en el trabajo;
el \cref{sec:conclusiones} muestra las conclusiones, limitaciones y posibles continuaciones de este trabajo.


    (Desarrollo)	 (Times New Roman, 12, negrita) No hay una sección que se denomine “Desarrollo”. En esta sección  se responde a la pregunta: “¿cómo se ha hecho el trabajo?". Esta es la sección más importante del trabajo debido a que se describe, en detalle, la propuesta presentada. (Times New Roman, 12).


    \section{Trabajos Relacionados}\label{sec:trabajos_relacionados}

Si bien no se encontró un caso con características tan similares, si se encontraron casos de sistemas informáticos adoptando motores de reglas para facilitar su adaptación a cambios en las reglamentaciones. 
A continuación describimos brevemente algunos de ellos.

En \cite{medic2019calculation} se reporta un uso similar de un \acrshort{brms} \cite{proctor2012drools} para el cálculo de los precios de las distintas pólizas de una aseguradora, siendo el principal objetivo buscado la mantenibilidad y la adaptabilidad a cambios en las regulaciones.

En \cite{sampol2019sistema}, en el contexto nacional, se describe el uso de un \acrshort{brms} en aplicaciones desarrolladas para SENASA con el objetivo de lograr que tengan mayor flexibilidad frente a cambios en las reglas de negocio que rigen los proceso que soportan.

También se detectó que en sistemas informáticos para organismos de salud privados, se argumenta flexibilidad en la especificación de reglas de negocio, pero sin mayor detalle que el comercial.


    Conclusión y Trabajos Futuros (Times New Roman, 12, negrita) 
Esta sección debe explicitar las limitaciones del trabajo presentado y establecer una discusión sobre los resultados o conclusiones presentadas. Se debe realizar un análisis de los aportes del trabajo frente a otros anteriores si los hubiera. Además, se deben establecer cuestiones abiertas y probables líneas adicionales en el marco de los resultados obtenidos. Los trabajos futuros se deben relacionar con la superación de las limitaciones del trabajo presentado. Muchas veces es, junto con el título, la parte más leída y por lo tanto debe ser de fácil comprensión. (Times New Roman, 12).


    \begin{thanks}
Agradecimientos (Times New Roman, 10, negrita) 
Si existiera, mencionarlos en forma concisa. Será escrito en fuente (Times New Roman, 10).
\end{thanks}


    % TODO: referencias

    Datos de Contacto: (Times New Roman, 10, negrita) 
Nombre y Apellido. Institución. Dirección postal. E-mail. Serán escritos en fuente (Times New Roman, 10, Cursiva)

\end{multicols}

\end{document}

\documentclass[12pt,a4paper,spanish]{article} 
\usepackage[spanish]{babel}

\setlength{\parindent}{0pt}
\usepackage{sectsty}
\sectionfont{\rmfamily\fontsize{12pt}{14.4pt}\selectfont}
\usepackage{titlesec}
\titlespacing*{\section} {0pt}{0ex}{0ex}
\pagestyle{empty}

\usepackage[bookmarks=true]{hyperref}

\usepackage{ragged2e}
\usepackage[none]{hyphenat}

% Agrega toc al toc
\usepackage{tocbibind}

\usepackage[acronym]{glossaries}
\makeglossaries

\input{acronyms}

\usepackage[inline]{enumitem}

\usepackage{cleveref}
\crefname{listing}{listado}{listados}
\Crefname{listing}{Listado}{Listados}

\usepackage{multicol}

\usepackage{fontspec}
\setmonofont{MesloLGS NF}
\setmainfont{Times New Roman}

% \usepackage{tikz}
% \usetikzlibrary{arrows.meta, positioning}

\usepackage[margin=2.5cm]{geometry}

% \usepackage{minted}
% \setminted[]{
%     breaklines,
%     breakafter=(.,
%     fontsize=\scriptsize,
%     escapeinside=||,
%     frame=single,
%     rulecolor=\color{black},
%     tabsize=2,
% }
% \setmintedinline[]{
%     fontsize=\scriptsize,
% }

% \usepackage{graphicx}
% \graphicspath{ {./images/} }

% \usepackage{csquotes}
\usepackage[backend=biber, style=alphabetic, sorting=none, bibstyle=alphabetic, maxnames=10, minnames=6,backref=true, autocite=inline, labelalpha=true, doi=false, url=false]{biblatex}
\addbibresource{the.bib}

% Use more than one optional parameter in a new commands
\usepackage{xargs}                      

% \usepackage[spanish,colorinlistoftodos,prependcaption,textsize=tiny]{todonotes}
% \newcommandx{\desarrollar}[2][1=]{\todo[inline,linecolor=blue,backgroundcolor=blue!25,bordercolor=blue,#1]{#2}}
% \newcommandx{\modificar}[2][1=]{\todo[linecolor=Plum,backgroundcolor=Plum!25,bordercolor=Plum,#1]{#2}}
% \newcommandx{\cambiarpor}[2][1=]{\todo[inline,linecolor=Plum,backgroundcolor=Plum!25,bordercolor=Plum,#1]{Cambiar $\uparrow$ por:\\#2}}
% \newcommandx{\revisar}[2][1=]{\todo[linecolor=red,backgroundcolor=red!25,bordercolor=red, inline,#1]{#2}}
% \newcommandx{\agregaren}[2][1=]{\todo[inline,linecolor=Plum,backgroundcolor=Plum!25,bordercolor=Plum,#1]{Agregar punto seguido $\uparrow$:\\#2}}
% \newcommandx{\agregar}[2][1=]{\todo[inline,linecolor=Plum,backgroundcolor=Plum!25,bordercolor=Plum,#1]{Agregar:\\#2}}
% \newcommandx{\info}[2][1=]{\todo[inline,linecolor=OliveGreen,backgroundcolor=OliveGreen!25,bordercolor=OliveGreen,#1]{#2}}
% \newcommandx{\eliminar}[2][1=]{\todo[inline,linecolor=red,backgroundcolor=red!25,bordercolor=red,#1]{\textbf{Eliminar $\uparrow$:}\\#2}}

% \newcommandx{\improvement}[2][1=]{\todo[linecolor=Plum,backgroundcolor=Plum!25,bordercolor=Plum,#1]{#2}}
% \newcommandx{\thiswillnotshow}[2][1=]{\todo[disable,#1]{#2}}

\newcommand{\SIOSU}{\acrshort{si}--\acrshort{osu}}
\newcommand{\SIDOSPU}{\acrshort{si}--\acrshort{dospu}}

\newcommand*{\thistitle}{
\bgroup
	\centering
    {\fontsize{16pt}{17.2pt}\selectfont\bfseries Parametrizando un Sistema Informático para Obra Social Universitaria con un Motor de Reglas \par}
	\vspace{0.5cm}
    {\fontsize{14pt}{16.8pt}\selectfont\bfseries Iván Brocas, Mg. Alejandro Sánchez, Dr. Carlos Humberto Salgado \par}
	{\fontsize{12pt}{14.4pt}\selectfont\bfseries\itshape Universidad Nacional de San Luis, Facultad de Ciencias Físico Matemática y Naturales \par}
	\vspace{0.2cm}
\egroup
}

\newcommand{\code}[1]{{\mintinline{java}{#1}}}
\newcommand{\scode}[1]{{\small\code{#1}}}

\let\originalacrfull\acrfull
\RenewDocumentCommand{\acrfull}{m}{%
  \ifglshasfield{user1}{#1}{%
    \glsentrylong{#1} (\glsentryuseri{#1}, \glsentryshort{#1})%
  }{%
    \originalacrfull{#1}%
  }%
}

\newcommand{\CenteredGraphic}[2]{%
	{\centering \resizebox{#2\textwidth}{!}{\includegraphics{#1}} \\}
}

\defbibheading{bibliography}[\refname]{%
  \section*{\fontsize{10pt}{12pt}\selectfont #1}%
}
\renewcommand*{\bibfont}{\fontsize{10pt}{12pt}\selectfont}



\renewenvironment{abstract}{
  \fontsize{10pt}{12pt}\selectfont
  {\bfseries Abstract}

  \itshape
}{
 \par\vspace{1em}
}

\newenvironment{keywords}{
  \fontsize{10pt}{12pt}\selectfont
  {\bfseries Palabras Clave}

}{
 \par\vspace{1em}
}


\title{Parametrizando un Sistema Informático para Obra Social Universitaria con un Motor de Reglas}
\author{Iván Brocas}

\begin{document}

\thistitle

\setlength\columnsep{1cm}
\begin{multicols}{2}
    \justifying
    \begin{abstract}
Esta publicación describe la parametrización del sistema informático de una obra social universitaria, en donde se separan las reglas de negocio del código fuente utilizando el motor de reglas OpenL Tablets.
%    
El objetivo es lograr un sistema más mantenible que pueda responder a modificaciones en las reglamentaciones que rigen su funcionamiento, pudiendo aplicar cambios a las reglas sin necesidad de cambiar el código fuente y volver a desplegarlo. 
%
En un contexto socioeconómico donde la única constante es el cambio, 
esto facilitará conservar el valor que el sistema brinda a la organización. 
%
Se hace foco en las reglas que rigen el cálculo de las cuotas de los afiliados, por ser estas complejas en la organización estudiada y susceptibles a cambios en la realidad socioeconómica del país.
%
La publicación reporta el trabajo realizado durante un proyecto integrador para la finalización de la {\CARRERA}.

\end{abstract}


    \begin{keywords}
    Obra social universitaria, motor de reglas, sistema informático, OpenL Tablets, cálculo de cuotas
    \end{keywords}

    \section{Introducción}\label{sec:intro}

Este proyecto integrador final se encuentra enmarcado en el contexto del desarrollo de un \acrfull{si} para \acrfullpl{osu} caracterizado por la centricidad en el afiliado y la agilidad. Este trabajo se centra en la segunda, procurando facilitar la adaptación del sistema en respuesta a cambios a un costo accesible y en tiempos mínimos a partir de la introducción de un motor de reglas que permita separar el código fuente de reglas del negocio tendientes a cambiar.

\subsection{Motivación}\label{ssec:motivacion}

El \acrshort{si} de una \acrshort{osu} debe mantenerse ``vivo''. 
Su utilidad, el valor que provee su funcionamiento, disminuye conforme pierde sintonía con los cambios que tanto la organización (por ejemplo, cambios en políticas y reglamentos locales) como su contexto sufren (como ser, cambios en leyes nacionales y provinciales, situaciones sociales o económicas a las que se debe atender, o eventos como Covid-19).
Dado que el cambio es la norma, no la excepción, la falta sistemática de evolución de un {\SIOSU} representa su agonía y eventual muerte, la cual es generalmente acompañada de numerosos perjuicios para la \acrshort{osu} y sus afiliados.

Al encontrarse con un sistema desactualizado, los funcionarios inicialmente suelen suplir las falencias del mismo por medio de trabajo manual. Conforme la brecha entre las reglas de negocio, dictadas por las reglamentaciones, y la implementación de las mismas en el \acrshort{si} crece, también lo hace la porción manual del trabajo. Esto resulta en más trabajo para el personal y trámites más lentos, lo que causa mayor descontento de los afiliados. De la misma forma, el decremento en la porción del trabajo realizada por el sistema se traduce en menos información para el respaldo de decisiones.

Las opciones que tienen las \acrshort{osu} para sacar a sus \acrshortpl{si} del camino de la obsolescencia pueden, en general, clasificarse en:

\begin{enumerate}
    \item [(a)] incorporar/escalar su propia área de informática de manera que esté a la altura del desarrollo y mantenimiento de software necesarios;
    \item [(b)] contratar a una empresa para que esté a cargo de este desarrollo y mantenimiento; o
    \item [(c)] comprarle a una empresa el producto ya desarrollado y pagar por su adaptación y mantenimiento 
\end{enumerate} 

Es preciso señalar que las opciones (a) y (b) requieren que la \acrshort{osu} sea propietaria del sistema a revitalizar.

El reemplazo del sistema agonizante por uno propietario ``enlatado'', la opción (c), a primera vista puede parecer la opción menos onerosa y más rápida de concretar. 
Sin embargo, un mayor poder de deción por parte de los afiliados, junto con un menor tamaño de la \acrshortpl{osu} (que suele implicar mayor complejidad en los convenios con proveedores) resultan en una mayor complejidad en comparación con su contraparte privada. En consecuencia, la opción (c) conlleva no solamente el pago de la licencia del \acrshort{si}, sino también el tiempo y coste de la adapación del mismo a las reglamentaciones del \acrshort{osu}.

La \acrfull{dospu} de la \acrfull{unsl} no escapa a este problema. 
La presidencia de \acrshort{dospu}, el Rectorado, la Facultad de Ciencias Físico-Matemáticas y Naturales, y su Departamento de Informática se encuentran colaborando para reemplazar su antiguo \acrshort{si} que ha fallado en evolucionar para acomodar los cambios que se han producido, y en el proceso causa perjuicios – como los antes enumerados – a la organización. 
%
El backend del nuevo sistema, \acrshort{si} -\acrshort{dospu}, está implementado utilizando el lenguaje de programación Java, el framework Spring y la base de datos PostgresSQL; y el frontend está implementado utilizando el framework Angular. 

En un trabajo previo, documentado en \cite{Vela2024}, el foco estuvo sobre implementar el expendio de órdenes procurando la centricidad en el paciente.
La agilidad para responder a cambios en el contexto aún no se abordó hasta este trabajo.

Este proyecto integrador constituye un paso hacia un {\SIOSU} ágil, fácil de adaptar a cambios en el contexto socio-económico y en las reglamentaciones que regulan su funcionamiento.

\subsection{Objetivo general}

El objetivo es parametrizar el {\SIOSU} de \acrshort{dospu}, introduciendo un mecanismo que permita separar de las reglas de negocio de la organización de su \acrshort{si}. 
A partir de la experiencia obtenida en el desarrollo de {\SIDOSPU}, se detectó que la reglamentación para el cálculo de la cuota de afiliados es compleja y muy susceptible a cambios en la realidad socio-económica del país. 
Estas características convierten a dicha funcionalidad en candidata ideal para enfocar el esfuerzo y, por lo tanto, la misma se adopta como alcance para el presente trabajo.

El mecanismo típico para lograr esta separación es un motor de reglas. Este permite la ejecución de reglas de negocio, las cuales son especificadas fuera del código del sistema, utilizando un lenguaje o formato específico del motor que se utilice.
Asimismo, se suele hacer uso de un \acrfull{brms}, que consiste en un motor de reglas junto con herramientas para la gestión de las reglas de negocio.

\subsection{Objetivos específicos}

Del objetivo general se desprenden los siguientes objetivos específicos:
\begin{enumerate}
    \item 
    Extraer las reglas del negocio correspondientes al cálculo de la cuota de afiliación a partir de las reglamentaciones pertinentes y su materialización en el código fuente del {\SIDOSPU}.
    \item \label{obj:esp:intelegible}
    Expresar dichas reglas en un lenguaje que resulte entendible para el personal de la \acrlong{osu}
    \item \label{obj:esp:independiente}
    Permitir la gestión de las especificaciones obtenidas de forma independiente al \acrshort{si}.
    \item Reducir el esfuerzo requerido para materializar cambios en las reglas del negocio. Este esfuerzo no debe requerir modificar el código fuente del sistema, ni realizar el despliegue de una nueva versión.
\end{enumerate}

El logro de estos objetivos permitirá que cambios en las reglas de negocio puedan materializarse a través de la edición de especificaciones externas,  permitiendo la reducción en la incidencia a o posiblemente una solución a los problemas mencionados anteriormente.

\subsection{Organización}
El resto de este trabajo se encuentra dividido en:
% TODO: 


    \section{Metodología} \label{sec:metodologia}

El enfoque adoptado combina pasos de metodologías de desarrollo de software y de metodologías de investigación científica, poniendo en juego conocimiento y habilidades adquiridos en el cursado de la Ingeniería en Informática de la \acrshort{unsl} en cada tarea ejecutada.

\revisar{No me queda claro que sería lo de establecer oportunidades (de mejora, de clarificación?)}
Se partió de la lectura y estudio de las ordenanzas y documentos detallando aspectos relevantes de \acrshort{dospu} y las fórmulas utilizadas para el cálculo de las cuotas de sus afiliados, que fue complementada con reuniones con miembros del equipo de desarrollo {\SIDOSPU} para despejar dudas y/o establecer oportunidades. 
De igual forma, se indagó sobre las tecnologías utilizadas en el sistema y la materialización de las reglas de negocio implementadas en el mismo.

Se extendieron los casos de prueba ya utilizados en el ciclo de integración continua de {\SIDOSPU}.
Estos nuevos casos se definieron en base a la información obtenida, utilizando una combinación de \acrfull{avl} y casos Ad Hoc para capturar el comportamiento esperado en un nivel de granularidad más fina y poder comprobarlo en cada paso ejecutado.

Adicionalmente, durante el estudio del código fuente encargado del cálculo de las cuotas, se detectaron oportunidades de mejora. 
En consecuencia, como paso intermedio, para mejorar la calidad del código y tener un mejor punto de partida para la posterior comparación con el motor de reglas, se realizó un refactorizado de la parte del sistema correspondiente. 
% Esta refactorización consistió principalmente en el uso de patrones de programación funcional para el manejo de errores monádico.
Esta refactorización consistió principalmente en el uso de patrones de programación funcional para el manejo de errores, haciendo uso de mónadas.

Por otro lado, se relevaron motores de reglas que se ajusten a las necesidades de expresividad requeridas para la materialización de las reglas de negocio de una \acrshort{osu}, y a las tecnologías de implementación del {\SIDOSPU}. 
Partiendo de este relevamiento, se realizó la selección del motor que se consideró más apropiado.

Luego se incorporó el motor de reglas en el {\SIDOSPU}.
% Seguidamente, se reimplementó parte del sistema recurriendo al motor de reglas para separar la especificación de las reglas de negocio del código fuente.
Seguidamente, se realizó la especificación de las reglas de negocio utilizando el formato requerido por el motor de reglas.

Esta nueva versión del cálculo de las cuotas convive con la mejorada y con la original. 
Esto se consiguió definiéndolas como implementaciones de una única interfaz Java y múltiples perfiles de Spring. 
Esto facilitó el análisis de resultados obtenidos y verificaciones a partir de casos de prueba. 

% Para la comparación de entre las implementaciones se utilizó como métrica la cantidad de líneas de código.  
Para la comparación entre las implementaciones se utilizó como métrica la cantidad de líneas de código.  
En el caso de la implementación con el motor de reglas se utilizan en su lugar las filas de las tablas en las que se especifican las reglas.


    \section{Afiliaciones y cálculo de cuotas}

Según lo establecido en \cite{CSOrd53}, además del personal docente y no docente de la \acrshort{unsl}, otras personas que cumplan con las condiciones dictadas también pueden afiliarse a \acrshort{dospu}.

A raíz de esto, los afiliados se encuentran dividos en las categorías titular (\cref{sec:titular}), familiar (\cref{sec:familiar}) y voluntario adherente (\cref{sec:adherente}). Asimismo cada categoría está dividida en subcategorías, cada una utilizando distintas fórmulas o coeficientes en las mismas para el cálculo del aporte del afiliado. Las definciones de cada categoría y subcategoría se pueden encontrar en \cite{CSOrd53} art. 24.

\subsection{Titular} \label{sec:titular}
Dentro de la categoría titular se distingue entre obligatorio activo y voluntario jubilado.

\subsubsection{Obligatorio activo} 
Actualmente, este descuento no es cálculado por el \acrshort{si}, con lo cual está fuera del alcance de este trabajo.

\subsubsection{Voluntario jubilado} \label{sssec:jubilado}
Monto de la cuota (\cite{dospuRes21} art. 2 y Anexo I): 
\begin{displaymath}
0.02 * j_m + 0.05 * j_h
\end{displaymath}

, donde:
\begin{itemize}
    \item $j_m = \text{jubilación mínima}$
    \item $j_h = \text{haber jubilatorio}$
\end{itemize}

En caso de que dos voluntarios jubilados se tengan vínculo de cónyuge o conviviente, se aplica un descuento del 30\% a la cuota del afiliado (cálculada con la fórmula expuesta para esta subcategoría) con menor haber percibido, quedando como 70\% del monto calculado.

Adicionalmente, el valor de esta cuota no puede ser inferior a un valor de referencia (\scode{cuotaMinimaJubilado}).

Por último, en caso de no tener un haber percibido cargado en el sistema, se utiliza otro valor de referencia (\scode{cuotaMaximaJubilado}) como el monto de la misma.

\subsection{Familiar} \label{sec:familiar}
Los familiares de obligatorios activos son no aportantes.

Por otra parte, para cónyuges y convivientes de voluntarios jubilados, su cuota es un 70 \% de la del afiliado titular (\cite{dospuRes21} art. 2 y Anexo I).

\subsection{Voluntario adherente} \label{sec:adherente}
Dentro de esta categoría, se distingue entre las subcategorías: pensionado, becarios y personal ad honorem de la \acrshort{unsl}, agente \acrshort{unsl} con licencia, ascendientes en primer grado (de un afiliado titular), hijos (que hayan dejado de reunir las condiciones de no aportantes), familiares adherentes, universitarios adherentes, ex-afiliados a \acrshort{dospu}, agentes vinculados a \acrshort{dospu} y adherente de edad avanzada.

Según \cite{dospuRes21} art. 3 para los afiliados de esta categoría, a excepción de las subcategorías tratadas en los \crefrange{sssec:pensionado}{sssec:edad_avanzada}, el monto de la cuota se calcula como un porcentaje sobre el valor de referencia \acrshort{cmmu}. Los porcentajes se encuentran en el Anexo II de la referencia mencionada.

La \acrfull{cmmu} se define en el 6 \% del sueldo total bruto de un Profesor Universitario Titular Exclusivo con Máxima Antigüedad (\cite{dospuRes21} art. 3).

\subsubsection{Pensionado}\label{sssec:pensionado}
Monto de la cuota (\cite{dospuRes21} art. 2 y Anexo I):
\begin{displaymath}
0.02 * j_m + 0.05 * p
\end{displaymath}

, donde:
\begin{itemize}
    \item $j_m = \text{jubilación mínima}$
    \item $p = \text{pensión}$
\end{itemize}

\subsubsection{Ascendiente en primer grado}
Para un afiliado de esta categoría con más de diez (10) años de antigüedad en DOSPU, se utiliza como valor de referencia \acrshort{cmmu}20, equivalente a 6\% de un Profesor Universitario Titular Exclusivo con una Antigüedad correspondiente veinte años, en lugar de la \acrshort{cmmu}\cite{dospuRes60}.

Asimismo, para afiliados mayores a 66 años de edad, se toma el 150\% en lugar de 200\% del valor de referencia, siendo este último el utilizado para los ascendientes en primer grado sin la antigüedad requerida.

En caso de no contar con dichos aportes, el valor de la cuota se calcula como un porcentaje sobre el valor de referencia \acrshort{cmmu} (\cite{dospuRes21} art. 3).

\subsubsection{Adherentes de edad avanzada}\label{sssec:edad_avanzada}
El monto a abonar por los afiliados pertenecientes a esta subcategoría depende de si el afiliado tiene 25 o más años de aporte de \acrfull{decom} (\cite{dospuRes7} art. 1.d.):
\begin{itemize}
    \item En caso de tener dicho aporte se toma un 150\% de la \acrshort{cmmu}
    \item En caso contrario se toma un 200\% de la \acrshort{cmmu}.
\end{itemize}

\subsection{Aportes y seguros adicionales}
Adicionalmente, dependiendo de la categoría y subcategoría del afiliado, el monto de la cuota pued incluir aportes a \acrfull{fesac} y \acrfull{sumas}, así como un seguro en caso de fallecimiento \cite{dospuRes31, dospuRes43, dospuRes71}.

\subsection{Modificadores de afiliación}
Para afiliados voluntarios adherente que ingresen con enfermedades preexistentes o que excedan la edad de 65 años se aplica un modificador al monto de la cuota cálculada, el cual depende del carácter de la enferdad del afiliado (ver \cref{tbl:modificadores}).

\begin{table}
\begin{tabular}{|c|c|}
    \hline
    Carácter de la enfermedad & Modificador \\ \hline
    Temporario & 2 \\ \hline
    Crónico & 3 \\ \hline
    De mayor complejidad & 2 \\ \hline
\end{tabular}
\caption{Modificadores de afiliación}
\label{tbl:modificadores}
\end{table}


    \input{selccion}

    \section{Integración}
\label{sec:integracion}

Para la integración de OpenL Tablets con {\SIDOSPU} existen dos alternativas.

La primera consiste en exponer las reglas por medio de un servicio web utilizando OpenL Rule Services. Esto nos permite integrar con distintas aplicaciones, sobre distintas plataformas, utilizar varias fuentes de datos y exponer varios proyectos y/o módulos mediante un único servicio web.

La segunda alternativa consiste en incluir OpenL Tablets como biblioteca y generar clases wrapper.
Estas últimas se generan en tiempo de ejecución a partir del contenido de las tablas en documentos Excel, exponiendo las reglas definidas en los mismos como métodos.
La principal ventaja de esta opción es que resulta en un menor costo de comunicación, dado que se realiza por medio de llamadas a métodos entre clases Java.

Considerando que las reglas serán utilizadas únicamente por {\SIDOSPU}, y estarán en un único proyecto, no pudiéndose sacar partido de los beneficios de un servicio web, se decidió utilizar la segunda opción.
%
%
El diagrama en la \cref{fig:integration} muestra un esquema de la integración resultante.
Los rectángulos representan una o varias clases Java, siendo la comunicación entre las mismas por llamado de sus respectivos métodos.

\begin{figure*}
    \centering
    \begin{tikzpicture}[
            auto,
            inner sep=3mm,
            box/.style={draw, rectangle, align=center},
            alt-box/.style={draw, rectangle, align=center, rounded corners=12pt},
            pre/.style={Stealth-},
            post/.style={-Stealth},
            alt-pre/.style={dashed, Stealth-},
            alt-post/.style={dashed, -Stealth},
        ]
        \node[box] (system) {SI-DOSPU};
        \node[box, right=of system] (service) {CalculoReglasServiceImpl}
        edge[pre] (system);
        \node[box, below=0.5cm of service] (clases) {Clases Integración}
        (clases.west) edge[post] (system);
        \node[box, right=of service] (wrapper)  {Clase Wrapper}
        edge[pre] (service)
        edge[post] (clases.east);
        \node[alt-box, above=of wrapper] (rules)  {Reglas (excel)}
        edge[alt-post] node {\small Compilado a} (wrapper)
        (rules.west) edge[alt-pre] node[swap] {\small Lee} (service);
    \end{tikzpicture}
    \caption{Integración OpenL Tablets con SI-DOSPU}
    \label{fig:integration}
\end{figure*}


\subsection{Clases de integración}\label{ssec:integracion:clases}

Como se mencionó en el \cref{sec:motores}, OpenL Tablets permite hacer uso directo de objetos Java dentro de las reglas.
Asimismo, permite hacer uso de clases de forma directa.
Sin embargo, las clases del {\SIDOSPU} lidian con cuestiones no del todo relevantes para el cálculo de las cuotas, como el manejo de errores y el acceso a datos que involucra comunicación con varias clases.
Para evitar contaminar con estos aspectos las reglas de cálculo, se crearon clase que los abstraen:
\begin{itemize}
    \item \scode{Afiliacion} encapsula complejidades de acceder a datos de la afiliación, como la categoría, subcategoría, si tiene cónyuge, etc.;
    \item \scode{Valores}, de manera similar, abstrae el acceso a otros valores del sistema relevantes pare el cálculo, como por ejemplo, el \acrshort{cmmu}; y
    \item \scode{Numero}, para poder operar sobre valores de tipo \scode{BigDecimal} utilizando operadores tales como +, -, *, /, etc.
\end{itemize}

\subsection{Cuota de Voluntario adherente}

% OpenL Tablets ofrece una variedad de tipos de tablas con distintas utilidades, en este trabajo se hizo uso de tablas de configuración, búsqueda y decisión. 
% La tabla de decisión posee la mayor flexibilidad y es la utilizada para la mayor parte de la lógica implementada. 

Esta sección sirve dos propósitos.
Por un lado describe como se especifica el cálculo de la cuota para voluntarios adherentes, y por otro lado introduce la sintaxis de las reglas.

OpenL Tablets ofrece una variedad de tipos de tablas \cite{openl-decision-table}.
Aquí describiremos brevemente el formato de las tablas de decisión, el cual es suficiente para expresar la lógica de negocio del cálculo de la cuota de afiliación.
%
La tabla de decisión en el \cref{tbl:cambio:original} aborda el caso de los voluntarios adherentes.
A continuación describimos su formato.

\begin{table*}[h]
    \centering
    \includegraphics[width=1\textwidth]{voluntario.png}
    \caption{Cálculo de cuota de voluntario adherente}
    \label{tbl:cambio:original}
\end{table*}


% \begin{table*}[h]
%     \centering
%     \begin{tabular}{|p{7cm}|p{7cm}|}
%         \hline
%         \multicolumn{2}{|c|}{Rules Numero CalcularVoluntarioAdherente(Afiliacion afiliado, Valores valores)}                    \\ \hline
%         C1                                                              & RET1                                                \\ \hline
%         subcategoria                                                    &                                                     \\ \hline
%                                                                         & Numero monto                                        \\ \hline
%         subcategoria                                                    & Monto                                               \\ \hline
%         subcategoria-afiliacion-ascendiente-mas-diez-antiguedad         & =CMMU20 * CoeficienteVoluntario(subcategoria, edad) \\ \hline
%         subcategoria-afiliacion-conyuge-ascendiente-mas-diez-antiguedad &                                                     \\ \hline
%                                                                         & =CMMU * CoeficienteVoluntario(subcategoria, edad)   \\ \hline
%     \end{tabular}
%     \caption{Data from calcva.xls}
%     \label{tab:calcva}
% \end{table*}

% \begin{minted}{BNF}
% Rules <signatura>
% <signatura>::=<tipo> <nombre_tbl>(<pars>)
% <pars>::=<tipo> <nombre>
% \end{minted}


La fila 1 contiene el encabezado de la tabla.
En nuestro ejemplo: \tblcode{Rules} indica que la tabla contiene reglas,
\tblcode{Numero} es el tipo de retorno y \tblcode{CalcularVoluntarioAdherente(Afiliacion afiliado, Valores valores)} es el nombre de la tabla con sus dos parámetros.

La fila 2 define si la columna es una condición o valor de retorno, indicándose como  \tblcode{Cn} para la n-esima condición y \tblcode{RETn} para el n-esimo posible valor de retorno.
En el ejemplo, se indica que las columnas 1 y 2 corresponden a condiciones, y que en la 3 está el valor de retorno.

La fila 3 especifica la condición para cada columna en formato BEX \cite{openl-bex}.
Para \tblcode{C1} del ejemplo, se especifica \tblcode{subcategoria==valor}.
El lado izquierdo es inferido por nombre de uno de los atributos del parámetro \tblcode{afiliado} y el derecho se define en la fila 4.
La celda vacía para \tblcode{RET1} indica que no hay condiciones para ese retorno y se evalúa a \tblcode{Verdadero}.

% \begin{description}

    % \item[Fila 1: ] Contiene el encabezado de la tabla.
    %       En nuestro ejemplo: \tblcode{Rules} indica que la tabla contiene reglas,
    %       \tblcode{Numero} es el tipo de retorno y \tblcode{CalcularVoluntarioAdherente(Afiliacion afiliado, Valores valores)} es el nombre de la tabla con sus dos parámetros.

    % \item[Fila 2: ] Define si la/s columna/s es/son una condición o valor de retorno, indicándose como  \tblcode{Cn} para la n-esima condición y \tblcode{RET1} para el primer valor de retorno, respectivamente.
    %       En el ejemplo, se indica que las columnas 1 y 2 corresponden a condiciones, y que en la 3 está el valor de retorno.

    % \item[Fila 3: ] Especifica las condiciones para cada columna en formato BEX \cite{openl-bex}.
    %       Para \tblcode{C1} del ejemplo, se especifica \tblcode{subcategoria==valor}, donde el lado izquierdo es inferido por nombre de uno de los atributos del parámetro \tblcode{afiliado} y el derecho se define en la fila 4.
    %       La celda vacía para \tblcode{RET1} indica que no hay condiciones para ese retorno y se evalúa a \tblcode{Verdadero}.

    % \item[Fila 4: ] Define un parámetro por columna cuyos valores se definen a partir de la fila 6.
    %       Para \tblcode{C1} define de nombre \tblcode{valor} y tipo \tblcode{String}, y para \tblcode{RET1} el nombre \tblcode{monto} de tipo \tblcode{Numero}.

    % \item[Fila 5: ] Contiene nombres descriptivos para los parámetros, ignorados por el motor.

    % \item[Fila 6+:] Especifica los valores concretos para los parámetros.
    %       También pueden contener expresiones matemáticas o llamadas a otras reglas.

% \end{description}

La fila 4 define un parámetro por columna cuyos valores se definen a partir de la fila 6.
Para \tblcode{C1} define de nombre \tblcode{valor} y tipo \tblcode{String}, y para \tblcode{RET1} el nombre \tblcode{monto} de tipo \tblcode{Numero}.

La fila 5 contiene nombres descriptivos para los parámetros, ignorados por el motor.

Las filas 6, en adelante, especifican los valores concretos para los parámetros.
También pueden contener expresiones matemáticas o llamadas a otras reglas.
%
Para \tblcode{C1}, las filas 6 y 7 indican dos valores, que corresponden a ascendientes de primer grado con más de diez años de antigüedad y su cónyuge, respectivamente.
En ambos casos, el valor de \tblcode{RET1} es el mismo.
Se calcula como la multiplicación de un coeficiente, obtenido a partir de la tabla \tblcode{CoeficienteVoluntario} (no incluida en esta publicación), según la subcategoría y edad del afiliado, por el \acrshort{cmmu20}.
%
La fila 8 de \tblcode{C1} se encuentra vacía.
El motor interpreta que esta fila debe utilizarse por defecto para cualquier otro valor que no haya unificado aún.
El valor de retorno varía en este caso, ya que se utiliza el \acrshort{cmmu} en la multiplicación.

\subsection{Cuota de voluntario jubilado}

El cálculo del monto a abonar por jubilados es uno de los más complejos (junto con el caso de los voluntarios adherentes).
El mismo ocupa más de 150 líneas de código Java en la implementación original del \acrshort{si}.

Es cálculo inicia con el \cref{tbl:calculo:jubilado:1}.
Se debe notar un atajo utilizado en la notación para las expresiones de las dos condiciones, \tblcode{C1} y \tblcode{C2}.
Se coloca unicamente, el nombre del campo, que se obtiene del parámetro \tblcode{Afiliado}, cuando la expresión buscada es una igualdad con los valores que se colocaran en las celdas debajo.
Entonces, si el afiliado no tiene haber percibido actualizado, se retorna el valor calculado en el \cref{tbl:calculo:jubilado:sinhaber}. 
En caso de si tenerlo, el cálculo continua con el 
\cref{tbl:calculo:jubilado:conconyuge} o con el \cref{tbl:calculo:jubilado:sinconyuge}, dependiendo de si el afiliado tiene un cónyuge o conviviente que también sea titular en la obra social, o no.

\begin{table*}
    \centering
    \includegraphics[width=.93\textwidth]{jubilado.png}
    \caption{Cálculo de cuota de jubilado}
    \label{tbl:calculo:jubilado:1}
\end{table*}

El cálculo del monto para un afiliado sin haber actualizado se muestra en el \cref{tbl:calculo:jubilado:sinhaber}.
Se toma como monto la cuota máxima de jubilado por un modificador, que depende de si el afiliado tiene un cónyuge o conviviente como afiliado familiar.

\begin{table*}
    \centering
    \includegraphics[width=0.7\textwidth]{jubiladoSinHaber.png}
    \caption{Cálculo de cuota de jubilado sin haber actualizado}
    \label{tbl:calculo:jubilado:sinhaber}
\end{table*}


\begin{table*}
    \centering
    \includegraphics[width=1\textwidth]{jubiladoConConyuge.png}
    \caption{Cálculo de cuota de jubilado con cónyuge}
    \label{tbl:calculo:jubilado:conconyuge}
\end{table*}


\begin{table*}
    \centering
    \includegraphics[width=0.7\textwidth]{jubiladoSinConyuge.png}
    \caption{Cálculo de cuota de jubilado sin cónyuge}
    \label{tbl:calculo:jubilado:sinconyuge}
\end{table*}


\begin{table*}
    \centering
    \includegraphics[width=0.8\textwidth]{jubiladoBase.png}
    \includegraphics[width=0.8\textwidth]{jubiladoConConyugeResponsable.png}
    \caption{Cálculo cuota base jubilado}
    \label{tbl:calculo:jubilado}
\end{table*}

\subsection{Realizando un cambio}\label{ssec:integracion:cambio}
Para ilustrar las ventajas de estos cambios en el \SIDOSPU a la hora de realizar cambios en las reglas, consideremos la aplicación del siguiente cambio en el valor de las cuotas:

\emph{
    Para el cálculo de la cuota de un afiliado con categoría voluntario adherente y subcategoría agente \acrshort{unsl} con licencia, el monto de la cuota es equivalente al monto de los aportes y contribuciones 9\% del sueldo bruto que percibiría como si estuviera en actividad. De igual forma, dicho monto no puede ser inferior a los porcentajes de la \acrshort{cmmu} que se utilizan en el cálculo anteriormente presentado para la categoría y subcategoría.
}

Sin el uso del motor de reglas, este cambio implica el cambio/adición de una 32 líneas de código. Seguidamente, se debe volver a compilar y desplegar el sistema hacer estos cambios efectivos.

Por otra parte, haciendo uso del OpenL Tablets, con la integración descrite, se requiere realizar los cambios entre los cuadros \ref{tbl:cambio:original} y \ref{tbl:cambio:cambiado}. El cambio es detectado en tiempo de ejecución y se generan las nuevas clases wrapper, sin necesidad de volver a compilar o desplegar el sistema.



\begin{table*}
    \centering
    \includegraphics[width=0.8\textwidth]{voluntario_cambios.png}
    \caption{Cálculo modificado voluntario adherente modificado}
    \label{tbl:cambio:cambiado}
\end{table*}


    \input{implementacion}

    \section{Trabajos Relacionados}\label{sec:trabajos_relacionados}

Si bien no se encontró un caso con características tan similares, si se encontraron casos de sistemas informáticos adoptando motores de reglas para facilitar su adaptación a cambios en las reglamentaciones. 
A continuación describimos brevemente algunos de ellos.

En \cite{medic2019calculation} se reporta un uso similar de un \acrshort{brms} \cite{proctor2012drools} para el cálculo de los precios de las distintas pólizas de una aseguradora, siendo el principal objetivo buscado la mantenibilidad y la adaptabilidad a cambios en las regulaciones.

En \cite{sampol2019sistema}, en el contexto nacional, se describe el uso de un \acrshort{brms} en aplicaciones desarrolladas para SENASA con el objetivo de lograr que tengan mayor flexibilidad frente a cambios en las reglas de negocio que rigen los proceso que soportan.

También se detectó que en sistemas informáticos para organismos de salud privados, se argumenta flexibilidad en la especificación de reglas de negocio, pero sin mayor detalle que el comercial.


    \section{Conclusión y Trabajos Futuros}\label{sec:conclusiones}

Revisando los objetivos plantedos en el \cref{sec:intro}: en el \cref{sec:afiliaciones} se puede ver las reglas extraídas de la documentación mencionada en el \cref{sec:metodologia} (objetivo \ref{obj:esp:extraer}).

Adicionalmente, como se puede ver en el \cref{sec:resultados}, la introducción del motor de reglas permitió una reducción considerable en la cantidad de líneas necesarias para implementar el cálculo de las cuotas de afiliaciones. De esta reducción se infiere una mejora en la legibilidad (objetivo \ref{obj:esp:intelegible}).

Por otra parte, en el \cref{sec:integracion} se mencionó que estas reglas son almacenadas en documentos Excel, lo cual las separa de la base de código del sistema (objetivo \ref{obj:esp:independiente}), pudiendo gestionar estos documentos por separado. Asimismo, las clases wrapper utilizadas para acceder a las reglas pueden ser cambiadas en tiempo de ejecución, permitiendo realizar cambios sin la necesidad de volver a compilar o desplegar el sistema (objetivo \ref{obj:esp:esfuerzo}).

\subsection{Limitaciones}
Actualmente, las reglas deben ser modificadas mediante la edición de los documentos Excel que las contienen. Las herramientas para documentos con este formator carecen de herramientas para la validación y versionado de las reglas, ya que el formato que utizan no es nativo de Excel.

A su vez, queda pendiente una medición del impacto real en la reducción del esfuerzo necesario para la compresión de las reglas por el personal de la \acrshort{osu}.

\subsection{Trabajos Futuros}

\subsubsection{Continuar con la parámetrización del SI}
La continuación más natural para este trabajo es la parametrización de otras reglas de la \acrshort{osu}, utilizando las tecnologías descritas anteriormente, como por ejemplo el cálculo de las coberturas de las prácticas.

\subsubsection{Mejora en la gestión de reglas}
Para facilitar la gestión de reglas por parte del personal de \acrshort{dospu}, se podría hacer uso de OpenL Studio, configurandolo para que funcione en conjunto con el \acrshort{si} actual. OpenL Studio brinda un editor para las reglas, permitiendo la validación de las mismas en el editor, así como herramientas para el control de versiones y la restricción del acceso, permitiendo solo a personal con las credenciales adecuadas para que realice cambios.

\subsubsection{Estudio de la inteligibilidad}
Se podría realizar una estimación del esfuerzo necesario para la compresión de las reglas con el nuevo formato o una medición del tiempo necesario para que un miembro del personal de la \acrshort{osu} pueda realizar un cambio en las reglas de forma correcta, por ejemplo.


    Agradecimientos (Times New Roman, 10, negrita) 
Si existiera, mencionarlos en forma concisa. Será escrito en fuente (Times New Roman, 10).


    % TODO: referencias

    \begin{contact}
Iván Brocas. Universidad Nacional de San Luis. Avenida Ejército de los Andes 950, Ciudad de San Luis. ivanbrocas01@gmail.com
\end{contact}

\end{multicols}

\end{document}

\section{Enfoque adoptado}
La pregunta central de este trabajo, ímplícita en el \cref{sec:intro} es ¿Es
posible un {\SIOSU} ágil?, donde la agilidad es la capacidad de adaptarse con esfuerzo mínimo a cambios en las reglamentaciones de la organización que soporta. Responer a esta pregunta demanda una comprensión profunda de la situación actual.
Para esto se partió de la lectura y estudio de las ordenanzas y documentos detallando los aspectos necesarios de \acrshort{dospu} y las fórmulas utilizadas para el cálculo de las cuotas de sus afiliados, que fue complementada con reuniones con miembros del equipo de desarrollo {\SIDOSPU} para despejar dudas y/o establecer oportunidades. De igual forma, se indagó sobre las tecnologías utilizadas en el sistema y la materialización de las reglas de negocio implementadas en el mismo.

Con base en la información obtenida, se definieron pruebas adicionales para el \acrshort{si}, utilizando una combinación de \acrfull{avl} y casos Ad Hoc para capturar el comportamiento esperado por el sistema y porteriormente comprobarlo.

Adicionalmente, durante el estudio del código fuente encargado del cálculo de las cuotas, se detectaron patrones de programación menos qu ideales desde el punto de vista de la legibilidad. En consecuencia, como paso intermedio, para mejorar la calidad del código y tener un mejor punto de partida para la posterior comparación con el motor de reglas se realizó un refactorizado de la parte del sistema correspondiente.

Por otro lado, se relevaron motores de reglas que se ajusten a las necesidades de expresividad requeridas para la materialización de las reglas de negocio de una \acrshort{osu}, y a las tecnologías de implementación de {\SIDOSPU}. Partiendo de la información obtenida en este relevamiento, se realizó la selección del motor que se consideró más apropiado.

Luego se incorporó el motor de reglas en el {\SIDOSPU}. Segidamente, se reimplementó parte del sistema recurriendo al motor de reglas para separar la especificación de las reglas de negocio del código fuente.

Para facilitar la extracción de resultados, se extrajo la funcionalidad del cálculo de cuotas como una implementación de una Java (\scode{calculoCuotaService}). Del refactorizado del código y la integración del motor de reglas se generaron dos implementaciones adicionales de esta interfaz, las cuales coexisten en la base de código. Para la comparación de estas versiones se pueden utilizar dos métricas:
\begin{itemize}
    \item Líneas de código (cuantitativa): el número de líneas de código de las distintas versiones. En el caso de la implementación con el motor de reglas se utilizan en su lugar las filas de las tablas en las que se específican las reglas.
    \item Esfuerzo para la comprensión de las reglas (cualitativa): consiste en realizar una estimación del esfuerzo requerido por parte de un programador para la comprensión de las reglas, utilizando las especificaciones de las mismas y el código fuento como base para dicha estimación.
\end{itemize}

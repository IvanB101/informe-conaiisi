\section{Introducción}\label{sec:intro}

Este proyecto integrador final se encuentra enmarcado en el contexto del desarrollo de un \acrfull{si} para \acrfullpl{osu} caracterizado por la centricidad en el afiliado y la agilidad. 
Este trabajo se centra en la segunda, procurando reducir costos y tiempos de adaptación del sistema a cambios a partir de la introducción de un motor de reglas que permita separar del código fuente las reglas del negocio tendientes a cambiar.

El \acrshort{si} de una \acrshort{osu} debe mantenerse ``vivo''. 
El valor que provee su funcionamiento disminuye conforme pierde sintonía con los cambios que tanto la organización 
como su contexto sufren. 
Dado que el cambio es la norma, no la excepción, la falta sistemática de evolución de un {\SIOSU} representa su agonía y eventual muerte, la cual es generalmente acompañada de numerosos perjuicios para la \acrshort{osu}.

Los funcionarios, al encontrarse con un \acrshort{si} desactualizado, suelen responder por medio de trabajo manual. Conforme la brecha entre las reglas de negocio, dictadas por las reglamentaciones, y la implementación de las mismas en el \acrshort{si} crece, también lo hace la porción manual del trabajo. Esto resulta en más trabajo para el personal y trámites más lentos, lo que causa mayor descontento de los afiliados. De la misma forma, el decremento en la porción del trabajo realizada por el sistema se traduce en menos información para el respaldo de decisiones.

La \acrfull{dospu} de la \acrfull{unsl} no escapa a este problema. 
Su presidencia, el Rectorado, la Facultad de Ciencias Físico-Matemáticas y Naturales, y su Departamento de Informática se encuentran colaborando para reemplazar su antiguo \acrshort{si}, que ha fallado en evolucionar. 
%
El nuevo sistema, \acrshort{si}-\acrshort{dospu}, está siendo desarrollado usando prácticas de las metodologías ágiles: scrum, product discovery, behaviour driven development, test driven development, e integración continua.
%
Las tecnologías de implementación incluyen, el lenguaje de programación Java, el framework Spring \cite{springframework} y la base de datos PostgreSQL \cite{postgresql} para el backend, y Angular \cite{angular} para el frontend. 

En un trabajo previo, documentado en \cite{Vela2024}, el foco estuvo sobre implementar el expendio de órdenes procurando la centricidad en el paciente.
%
Este proyecto integrador aborda la agilidad requerida por el {\SIOSU}, entendida como la facilidad de adaptación a cambios en el contexto socio-económico y en las reglamentaciones que regulan su funcionamiento.

El objetivo es parametrizar {\SIDOSPU}, introduciendo un mecanismo que permita separar las reglas de negocio de la organización, de su \acrshort{si}. 
Se detectó que la reglamentación para el cálculo de la cuota de afiliados es compleja y muy susceptible a cambios en la realidad socio-económica del país. 
Esto se debe a que la obra social busca equidad en la distribución de la carga de los aportes, lo que lleva a plantear diversos tipos de afiliación y fórmulas de cálculo de cuota. 
Estas características convierten a dicha funcionalidad en candidata ideal para enfocar el esfuerzo.

El mecanismo típico para lograr esta separación es un motor de reglas. 
Este permite la ejecución de reglas de negocio, especificadas fuera del código del sistema, utilizando un lenguaje o formato específico.

Del objetivo se desprenden cuatro objetivos específicos:
\begin{enumerate}
    \item \label{obj:esp:extraer}
    Conocer las reglas del negocio correspondientes al cálculo de la cuota de afiliación a partir de las reglamentaciones pertinentes y su materialización en el código fuente del {\SIDOSPU}.
    \item \label{obj:esp:intelegible}
    Expresar dichas reglas en un lenguaje que resulte entendible para el personal de la \acrlong{osu}.
    \item \label{obj:esp:independiente}
    Permitir la gestión de las especificaciones obtenidas de forma independiente al \acrshort{si}.
    \item \label{obj:esp:esfuerzo}
    Reducir el esfuerzo requerido para materializar cambios en las reglas del negocio, sin requerir modificar el código fuente del sistema, ni realizar el despliegue de una nueva versión.
\end{enumerate}

El logro de estos objetivos permitirá que cambios en las reglas de negocio puedan materializarse a través de la edición de especificaciones externas,  permitiendo la reducción en la incidencia a problemas mencionados anteriormente.

\section{Trabajos Relacionados}\label{sec:trabajos_relacionados}

Si bien no se encontró un caso con características tan similares, si se encontraron casos de sistemas informáticos adoptando motores de reglas para facilitar su adaptación a cambios en las reglamentaciones. 
A continuación describimos brevemente algunos de ellos.

En \cite{medic2019calculation} se reporta un uso similar de un \acrshort{brms} \cite{proctor2012drools} para el cálculo de los precios de las distintas pólizas de una aseguradora, siendo el principal objetivo buscado la mantenibilidad y la adaptabilidad a cambios en las regulaciones.

En \cite{sampol2019sistema}, en el contexto nacional, se describe el uso de un \acrshort{brms} en aplicaciones desarrolladas para SENASA con el objetivo de lograr que tengan mayor flexibilidad frente a cambios en las reglas de negocio que rigen los proceso que soportan.

También se detectó que en sistemas informáticos para organismos de salud privados, se argumenta flexibilidad en la especificación de reglas de negocio, pero sin mayor detalle que el comercial.


El resto de este trabajo se encuentra dividido en:
el \cref{sec:metodologia} explica la metodología utilizada; 
el \cref{sec:motores} enumera los motores de reglas considerados y el razonamiento detrás de la selección realizada dentro de las opciones;
el \cref{sec:afiliaciones} trata las distintas categorías de afiliaciones de la \acrshort{osu} y el cálculo de sus cuotas;
el \cref{sec:integracion} explica cómo se realizó la integración del motor de reglas seleccionado con el {\SIOSU};
el \cref{sec:resultados} expone los resultados obtenidos en el trabajo;
el \cref{sec:conclusiones} muestra las conclusiones, limitaciones y posibles continuaciones de este trabajo.

\begin{abstract}
Este trabajo describe la parametrización de un sistema informático de una obra social universitaria. Esto consiste en la extracción de las reglas de negocio del código fuente del sistema, con el fin de separar la gestión de las mismas de la gestión del sistema. Con esto se busca lograr un sistema más mantenible que pueda responder rápidamente a modificaciones en las reglamentaciones que rigen su funcionamiento, al poder aplicar cambios a las reglas sin necesidad de cambiar el código fuente o volver a desplegar el sistema. En un contexto socioeconómico donde la única constante es el cambio esto facilitaría que el sistema se mantenga actualizado, evitando la obsolescencia y aumentando el valor que brinda a la organización. Este trabajo ce centra en las reglas que rigen el cálculo de las cuotas de los afiliados. Esto se debe a que la obra social estudiada busca equidad en la distribución de la carga de los ingresos económicos, lo que lleva a plantear diversos tipos de afiliación y fórmulas de cálculo de cuota asociadas. Para lograr la separación entre las reglas de negocio y el sistema informático, se hace uso de un motor de reglas, el cual es integrado con el sistema.
\end{abstract}

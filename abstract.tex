\begin{abstract}
Esta publicación describe la parametrización del sistema informático de una obra social universitaria, en donde se separan las reglas de negocio del código fuente utilizando el motor de reglas OpenL Tablets.
%    
El objetivo es lograr un sistema más mantenible que pueda responder a modificaciones en las reglamentaciones que rigen su funcionamiento, pudiendo aplicar cambios a las reglas sin necesidad de cambiar el código fuente y volver a desplegarlo. 
%
En un contexto socioeconómico donde la única constante es el cambio, 
esto facilitará conservar el valor que el sistema brinda a la organización. 
%
Se hace foco en las reglas que rigen el cálculo de las cuotas de los afiliados, por ser éstas complejas en la organización estudiada y susceptibles a cambios en la realidad socioeconómica del país.
%
La publicación reporta el trabajo realizado durante un proyecto integrador para la finalización de la {\CARRERA}.

\end{abstract}

\begin{abstract}
Esta publicación describe la parametrización de un sistema informático de una obra social universitaria, en donde se extraen las reglas de negocio del código fuente.
%    
% Este trabajo describe la parametrización de un sistema informático de una obra social universitaria. 
% Esto consiste en la extracción de las reglas de negocio del código fuente del sistema, con el fin de separar la gestión de las mismas.
%
El objetivo es lograr un sistema más mantenible que pueda responder a modificaciones en las reglamentaciones que rigen su funcionamiento, pudiendo aplicar cambios a las reglas sin necesidad de cambiar el código fuente y/o volver a desplegar el sistema. 
%
En un contexto socioeconómico donde la única constante es el cambio, esto facilitaría que el sistema se mantenga actualizado, evitando la obsolescencia y conservando el valor que brinda a la organización. 
Se hace foco en las reglas que rigen el cálculo de las cuotas de los afiliados. 
Esto se debe a que la obra social estudiada busca equidad en la distribución de la carga de los aportes, lo que lleva a plantear diversos tipos de afiliación y fórmulas de cálculo de cuota. 
La separación entre las reglas de negocio y el sistema informático se logra integrando un motor de reglas al sistema.
\end{abstract}
